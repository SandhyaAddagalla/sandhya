\documentclass[12pt]{article}
\usepackage{graphicx}
\usepackage{enumerate}
\usepackage{amsmath}
\usepackage{ragged2e}
\usepackage{listings}
\newenvironment{matchtabular}{%
  \setcounter{matchleft}{0}%
  \setcounter{matchright}{0}%
  \tabularx{\textwidth}{%
    >{\leavevmode\hbox to 1.5em{\stepcounter{matchleft}\arabic{matchleft}.}}X%
    >{\leavevmode\hbox to 1.5em{\stepcounter{matchright}\alph{matchright})}}X%
    }%
}{\endtabularx}

%\usepackage{blindtext}
\usepackage{multicol}
\title{Multicols Demo}
\setlength{\columnsep}{3cm}
\usepackage{tabularx}
 \usepackage[utf8]{inputenc}
\newcounter{matchleft}
\newcounter{matchright}

\newcommand*{\vv}[1]{\vec}
\newcommand{\myvec}[1]{\ensuremath{\begin{pmatrix}#1\end{pmatrix}}}
\newcommand{\mydet}[1]{\ensuremath{\begin{vmatrix}#1\end{vmatrix}}}
\providecommand{\brak}[1]{\ensuremath{\left(#1\right)}}
\providecommand{\lbrak}[1]{\ensuremath{\left(#1\right.}}
\providecommand{\rbrak}[1]{\ensuremath{\left(#1\right)}}
\providecommand{\sbrak}[1]{\ensuremath{{}\left[#1\right]}}
%\let\vec\mathbf

\begin{document}
\begin{center}
\textbf\large{CHAPTER-8 \\ Circle}

\end{center}
\section*{Section-A    [JEE Advanced/IIT-JEE]}
\section*{A    :  Fill in the Blanks}
\begin{enumerate}
\item If A and B are points in the plane such that $\frac{PA}{PB}=K$(constant) for all P on a given circle, then the value of k cannot be equal to.......(1982)
\item The points of intersection of the line $4x-3y-10=0$ and the circle $x^2+y^2-2x+4y-20=0$ are..... and.....(1983)
\item The lines $3x-4y +4=0$ and $6x-8y-7=0$ are tangents to the same circle. The radius of this circle is......(1984)
\item Let $x^2+y^2-4x-2y-11=0$ be a circle. A pair of tangents from the point (4, 5) with a pair of radii form a quadrilaterall of area .......(1985)
\item From the origin chords are drawm to the circle$(x- 1)^2+y^2=1$. The equation of the locus of the mid-points of these chords is .......(1985)
\item The equation of the line passing through the points of intersection of the circles $3x^2+3y^2-2x+12y-9=0$ and $x^2+y^2+6x+2y-15=0$ is .........(1986)
\item From the point A(0, 3) on the circle $x^2+4x+(y-3)^2=0$, a chord AB is drawn and extended to a point M such that AM=2AB. The equation of the locus of M is.......(1986)
\item The area of the triangle formed by the tangents from the point (4, 3) to the the circle $x^2+y^2=9$ and the line joining their points of contact is.......(1987)
\item If the circle C:$x^2+y^2=16$ intersects anothere circle $C_2$ of radius 5 in such manner that common chord is of maximum length and has a slope equal to $\frac{3}{4}$,then the coordinates of the center of $C_2$ are........(1988)
\item The area of the triangle formed by the positive x-axis and the normal and the tangent to the circle $x^2+y^2=4$ at $(1,\sqrt{3})$ is....... (1989)
\item If a circle passes through the points of intersection of the coordinate axes with the lines $\lambda x-y+1=0$ and $x-2y+1=0$, then the value of $\lambda$=........(1991)
\item The equation of the locus of the mid-points of the circle $4x^2+4y^2-2x+4y+1=0$  that subtend an angle of $\frac{2\pi}{3}$ at its centre is......(1993)
\item The intercept on the line $y=xby$ the circle $x^2+y^2-2x=0$ is AB. Equation of the circle with AB as a diameter is......(1996)
\item For each natural number k, let $C_k$, denote the circle with radius k centimetres and centre at the origin. On the circle $C_k$, $\alpha$-particle moves k centimetres in the counter-clockwise direction. After completing its motion on $C_k$ the particle moves to $C_k+1$ in the radial direction. The motion of the particle continues in this manner. The particle starts at (1,0). If the particle crosses the positive direction of the x-axis for the first time on the circle $C_n$. then n=.......(1997)
\item The chords of contact of the pair of tangents drawn from each point on the line $2x+y=4$ to circle$x^2+y^2=1$ pass through the point.....(1997)


\end{enumerate}

\section*{B    :    True/False}
\begin{enumerate}
\item No tangent can be drawn from the point $(\frac{5}{2}, 1)$ to the circumcircle of the triangle with vertices $(1,\sqrt{3})$,$(1,-\sqrt{3})$, $(3,-\sqrt{3})$.(1985)
\item The line $x+3y=0$ is a diameter of the circle $x^2+y^2-6x+2y=0$. (1989)

\end{enumerate}

\section*{C  :   MCQ'S with One Correct Answer}

\begin{enumerate}
\item A square is inscribed in the circle $x^2+y^2-2x+4y+3 =0$. Its sides are parallel to the coordinate axes. The one vertex of the square is (1980)
\begin{enumerate}
\item $(1+\sqrt{2},-2)$
\item $(1-\sqrt{2},-2)$
\item $(1,-2+\sqrt{2})$
\item none of these
\end{enumerate}
\item Two circles $x^2+y^2=6$ and $x^2+y^2-6x+8=0$ are given. Then the equation of the circle through their points of intersection and the point (1,1) is 1980)
\begin{enumerate}
\item $x^2+y^2-6x+4=0$
\item $x^2+y^2-3x+1=0$
\item $x^2+y^2-4y+2=0$
\item none of these
\end{enumerate}
\item The centre ofthe circle passing through the point (0, 1) and touching the curve $y =x^2$ at (2, 4) is (1983)
\begin{enumerate}
\item $\sbrak{\frac{-16}{5},\frac{27}{10}}$
\item $\sbrak{\frac{-16}{7},\frac{53}{10}}$
\item $\sbrak{\frac{-16}{5},\frac{53}{10}}$
\item none of these
\end{enumerate}
\item The equation of the circle passing through (1, 1) and the points of intersection of $x^2+y^2+13x-3y=0$ and $2x^2+2y^2+4x-7y-25=0$ is (1983)
\begin{enumerate}
\item $4x^2+4y^2-30x-10y-25=0$
\item $4x^2+4y^2+30x-13y-25=0$
\item $4x^2+4y^2-17x-10y+25=0$
\item none of these
\end{enumerate}
\item The locus of the mid-point of a chord of the circle $x^2+y^2=4$ which subtends a right angle at the origin is (1984)
\begin{enumerate}
\item $x+y=2$
\item $x^2+y^2=1$
\item $x^2+y^2=2$
\item $x+y=1$
\end{enumerate}
\item  Ifa circle passes through the point (a, b) and cuts the circle $x^2+y^2=k$ orthogonally, then the equation of the locus of ()
\begin{enumerate}
\item $2ax+2by-(a^2+b^2+k^2)=0$
\item $2ax+2by-(a^2-b^2+k^2)=0$
\item $x^2+y^2-3ax-4by+(a^2+b^2-k^2)=0$
\item $x^2+y^2-2ax-3by+(a^2+b^2-k^2)=0$
\end{enumerate}
\item Ifthe two circles $(x- 1)^2+(y-3)^2=r^2$ and $x^2+y^2-8x+2y+8=0$ intersect in two distinct points, then (1989)
\begin{enumerate}
\item $2<r<8$
\item $r<2$
\item $r=2$
\item $r>2$
\end{enumerate}
\item The lines $2x-3y=5$ and $3x-4y=7$ are diameters of acircleof area 154 sq. units. Then the equation of this circle is (1989)
\begin{enumerate}
\item $x^2+y^2+2x-2y=62$
\item $x^2+y^2+2x-2y=47$
\item $x^2+y^2-2x+2y=47$
\item $x^2+y^2-2x+2y=62$
\end{enumerate}
\item The centre of a circle passing through the points (0, 0),(1,0) and touching the circle $x2+y2=9$ is (1992)
\begin{enumerate}
\item $\sbrak{\frac{3}{2},\frac{1}{2}}$
\item $\sbrak{\frac{1}{2},\frac{3}{2}}$
\item $\sbrak{\frac{1}{2},\frac{1}{2}}$
\item $\sbrak{\frac{3}{2},-2^\frac{1}{2}}$
\end{enumerate}
\item The locus of the centre of a circle, which touches externally the circle $x^2+y^2-6x-6y+14=0$ and also touches the y-axis, is given by the equation: (1993)
\begin{enumerate}
\item  $x^2-6x-10y+14=0$
\item  $x^2-10x-6y+14=0$
\item  $y^2-6x-10y+14=0$
\item  $y^2-10x-6y+14=0$
\end{enumerate}
\item The circles $x^2+y^2-10x+16=0$ and $x^2+y^2=r^2$ intersect each other in two distinct points if (1994)
\begin{enumerate}
\item $r<2$
\item $r>8$
\item $2<r<8$
\item $2\le r\le 8$
\end{enumerate}
\item The angle between a pair of tangents drawn from a point p to the circle $x^2+y^2+4x-6y+9\sin^2\alpha+13\cos^2\alpha=0$ is  $2\alpha$. The equation of the locus of the point P is (1996)
\begin{enumerate}
\item $x^2+y^2+4x-6y+4=0$
\item $x^2+y^2+4x-6y-9=0$
\item $x^2+y^2+4x-6y+9=0$
\item $x^2+y^2+4x-6y-4=0$
\end{enumerate}
\item If two distinct chords, drawn from the point (p,q) on the circle $x^2+y^2=px+qy$ (where $pq\neq 0$) are bisected by the X-axis, then (1999)
\begin{enumerate}
\item $p^2=q^2$
\item $p^2=8q^2$
\item $p^2<8q^2$
\item $p^2>8q^2$
\end{enumerate}
\item The triangle PQR is inscribed in the circle $x^2+y^2=25$. If Q and R have co-ordinates (3,4) and (4,3) respectively, then $\angle PQR$ is equal to (2000)
\begin{enumerate}
\item $\frac{\pi}{2}$
\item $\frac{\pi}{3}$
\item $\frac{\pi}{4}$
\item $\frac{\pi}{6}$
\end{enumerate}
\item If the circles $x^2+y^2+2x+2khy+6=0,x^2+y^2+2ky+k=0$ intersect orthogonally, then k is
\begin{enumerate}
\item 2 or -$\frac{3}{2}$
\item -2 or -$\frac{3}{2}$
\item 2 or $\frac{3}{2}$
\item -2 or $\frac{3}{2}$
\end{enumerate}
\item  Let AB be a chord of the circle $x^2+y^2=r^2$ subtending a right angle at the centre. Then the locus of the centroid of the triangle PAB as P moves on the circle is (2001)
\begin{enumerate}
\item  a parabola
\item a circle
\item a ellipse
\item a pair of straight lines
\end{enumerate}
\item Let PQ and RS be tangents at the extremities of the diameter PR of a circle of radius r. If PS and RQ intersect at a point X on the circumference of the circle, then 2r equals (2001)
\begin{enumerate}
\item  $\sqrt{PQRS}$
\item $\frac{PQ+RS}{2}$
\item $\frac{2PQ\cdot RS}{PQ+RS}$
\item $\frac{\sqrt{PQ^2+RS^2}}{2}$
\end{enumerate}
\item Ifthe tangent at the point P on the circle $x^2+y^2+6x+6y=2$ meets a straight line $5x-2y+6=0$ at a point Q on the y- axis,then the length of PQ is (2002)
\begin{enumerate}
\item 4
\item $\sqrt[2]{5}$
\item 5
\item $\sqrt[3]{5}$
\end{enumerate}
\item The centre of circle inscribed in square formed by the lines $x^2-8x+12=0$ and$y^2 -14y+45=0$, is (2003)
\begin{enumerate}
\item (4,7)
\item (7,4)
\item (9,4)
\item (4,9)
\end{enumerate}
\item If one of the diameters of the circle $x^2+y^2-2x-6y+6=0$ is a chord to the circle with centre (2, 1),then the radius of the circle is (2004)
\begin{enumerate}
\item $\sqrt{3}$
\item $\sqrt{2}$
\item 3
\item 2
\end{enumerate}
\item A circle is given by $x^2+(y-1)^2=1$, another circle C touches it externally and also the x-axis, then the locus of its centre is (2005)
\begin{enumerate}
\item $\{(x,y):x^2=4y\}\cup\{(x,y):y\leq 0\}$
\item $\{(x,y):x^2+(y-1)^2=4\}\cup\{(x,y):y\leq 0\}$
\item $\{(x,y):x^2=y\}\cup\{(0,y):y\leq 0\}$
\item $\{(x,y):x^2=4y\}\cup\{(0,y):y\leq 0\}$
\end{enumerate}
\item Tangents drawn from the point P(1,8) to the circle $x^2+y^2-6x-4y-11=0$ touch the circle at the points A and B. The equation of the circumcircle of the triangle PAB is (2009)
\begin{enumerate}
\item $x^2+y^2+4x-6y+19=0$
\item $x^2+y^2-4x-10y+19=0$
\item $x^2+y^2-2x+6y-29=0$
\item $x^2+y^2-6x-4y+19=0$
\end{enumerate}
\item The circle passing through the point (-1,0) and touchingthe y-axis at (0, 2) also passes through the point (2011)
\begin{enumerate}
\item  $\sbrak{-\frac{3}{2},0}$
\item  $\sbrak{-\frac{5}{2},2}$
\item  $\sbrak{-\frac{3}{2},\frac{5}{2}}$
\item  $\sbrak{-4,0}$
\end{enumerate}
\item The locus of the mid-point of the chord of contact of tangents drawn from points lying on the straight line $4x-5y=20$ to the circle $x^2+y^2=9$ is (2012)
\begin{enumerate}
\item $20(x^2+y^2)-36x+45y=0$
\item $20(x^2+y^2)+36x-45y=0$
\item $36(x^2+y^2)-20x+45y=0$
\item $36(x^2+y^2)+20x-45y=0$
\end{enumerate}
\item A line $y=mx+1$ intersectrs the circle $(x-3)^2+(y+2)^2=25$ at the points P and Q. If the midpoint of the line segment PQ has x-coordinate $-\frac{3}{5}$, then which one of the following options is correct ? (2019)
\begin{enumerate}
\item $2\leq m<4$
\item $-3\leq m<-1$
\item $4\leq m<6$
\item $6\leq m<8$
\end{enumerate}
\end{enumerate}
\section*{D  :  MCQ'S with One or More Than One Correct Answer}

\begin{enumerate}
\item The equations of the tangents drawn from the origin to the circle $x^2+y^2-2rx-2hy+h^2=0$ are (1988)
\begin{enumerate}
\item x=0
\item y=0
\item $(h^2-r^2)x-2rhy=0$
\item $(h^2-r^2)x+2rhy=0$ 
\end{enumerate}
\item The number of common tangents to the circles $x^2+y^2=4$ and $x^2+y^2-6x-8y=24$ is (1998)
\begin{enumerate}
\item 0
\item 1
\item 3
\item 4
\end{enumerate}
\item If the circle $x^2+y^2=a^2$ intersects the hyperbola $xy=c^2$ in four points $P(x_1,y_1), Q(x_2 Y_2), R(x_3 ,y_3), S(x_4,y_4)$ then (1998)
\begin{enumerate}
\item $x_1+x_2+x_3+x_4=0$
\item $y_1+y_2+y_3+y_4=0$
\item $x_1x_2x_3x_4=c^4$
\item $y_1y_2y_3y_4=c^4$
\end{enumerate}
\item Circle(s) touching x-axis at a distance 3 from the origin and having an intercept of length $\sqrt[2]{7}$ on y-axis is (are) (2013)
\begin{enumerate}
\item $x^2+y^2-6x+8y+9=0$
\item $x^2+y^2-6x+7y+9=0$
\item $x^2+y^2-6x-8y+9=0$
\item $x^2+y^2-6x-7y+9=0$
\end{enumerate}
\item A circle S passes through the point (0, 1) and is orthogonal to the circles $(x- 1)^2+y^2=16$ and $x^2+y^2=1$. Then (2014)
\begin{enumerate}
\item radius of S is 8
\item radius of S is 7
\item Center of S is (-7,1)
\item Center of S is (-8,1)
\end{enumerate}
\item Let RS be the diameter of the circle $x^2+y^2=1$, where S is the point (1, 0). Let P be a variable point (other than R and S) on the circle and tangents to the circle at S and P meet at the point Q. The normal to the circle at P intersects a line drawn through Q parallel to RS at point E. Then the locus of E passes through the point(s) (2016)
\begin{enumerate}
\item $\frac{1}{3},\frac{1}{\sqrt{3}}$
\item $\frac{1}{4},\frac{1}{2}$
\item $\frac{1}{3},-\frac{1}{\sqrt{3}}$
\item $\frac{1}{4},\frac{1}{2}$ 
\end{enumerate}
\item Let Tbe the line passing through the points P(-2, 7) and Q(2.-5). Let $F_1$, be the set of allpairs of circles $(S_1, S_2)$ such that T is tangent to S, at P and tangent to $S_2$, at Q, and also such that $S_1$ and $S_2$ touch each other at a point, say, M. Let $E_1$, be the set representing the locus of M as the pair $(S_1, S_2)$ varies in $F_1$. Let the set of all straight line segments joining a pair of distinct points of $E_1$, and passing through the point R(1, 1) be $F_2$, Let $E_2$, be the set of the mid-points of the line segments in the set $F_2$ Then, which of the following statement(s) is (are) TRUE? (2018)
\begin{enumerate}
\item The point (2,7) lies in $E_1$
\item The point $(\frac{4}{5},\frac{7}{5})$ does NOT lies in $E_2$
\item The point $(\frac{1}{2},1)$ lies in $E_2$
\item  The point $(0,\frac{3}{2})$ does NOT lies in $E_1$
\end{enumerate}
\end{enumerate}
\section*{E  :  Subjective Problems}

\begin{enumerate}
\item Find the equation of the circle whose radius is 5 and which touches the circle $x^2+y^2-2x-4y-20=0$ at the point (5, 5).(1978)
\item Let A be the centre of the circlex $x^2+y^2-2x-4y-20=0$. Suppose that B(1,7) and D(4,-2) on the circle meet at the point C.(1981)
\item  Find1 the area of the quadrilateral ABCD. Find the equations of the circle passing through 4,3) and touching the lines x+y=2 and x-y =2. (1982)
\item Through a fixed point (h, k) secants are drawn to the circle $x^2+y^2=P$. Show that the locus of the mid-points of the secants intercepted by the circle is $x^2+y^2= hx +ky$. (1983)
\item The abscissa of the two points A and B are the roots of the equation $x^2+2ax-b^2=0$ and their ordinates are the roots of the equation $x^2+2px-q^2=0$. Find the equation and the radius of the circle with AB as diameter. (1984)
\item Lines $5x+12y-10=0$ and $5x-12y-40=0$ touch a circle $C_1$ of diameter 6. If the centre of $C_2$, lies in the first quadrant, find the equation of the circle $C_1$, which is concentric with $C_1$ and cuts intercepts of length 8 on these lines the tangents at the points (1986)
\item Let a given line $L_1$, intersects the x and y axes at P and Q, respectively. Let another line $L_2$, perpendicular to $L_1$, cut the x and y axes at R and S,respectively. Show that the locus of the point of intersection of the lines PS and OR is a circle passing through the origin. (1989)
\item The circlec $x^2+y^2-4x-4y+4=0$ is inscribed in a triangle which has two of its sides along the co-ordinate axes. The locus of the circumcentre of the triangle is
$x+y-xy+k(x^2+y^2)^{\frac{1}{2}}=0$. Find k. (1987)
\item If $\sbrak{m_i\frac{1}{m_i}},m_i>0,i=1,2,3,4$ are four distinct points on a circle ,then show that $ m_1m_2m_3m_4=1$ (1989)
\item A circle touches the line $y=x$ at a point P such that $OP=\sqrt[4]{2}$,, where O is the origin. The circle contains the point(-10,2) in its interior and the length of its chord on the line $x+y=0$ is $\sqrt[6]{2}$. Determine the equation of the circle. (1990)
\item Two circles, cach of radius 5 units, touch each other at (1,2).If the equation of their common tangent is $4x+3y=10$, find the cquation of the circles. (1991)
\item Let a circle be given by $2x(x-a)+y(2y-b)=0,(a\neq 0,b\neq 0))$.Find the condition on a and b if two chords,each biscected by the x- axis, can be drawn to the circle from $\sbrak{a,\frac{b}{2}}$. (1992)
\item Consider a family of circles passing through two fixed points A(3,7) and B(6,5). Show that the chords in which the circle $x^2+y^2-4x-6y-3=0$ cuts the members of the family are concurrent at a point. Find the coordinate of this point.
\item Find the coordinates of the point at which the circles $x^2+y^2-4x-2y=-4$ and $x^2+-12x-8y=-36$ touch each other. Also find equations common tangents touching the
circles in the distinct points.
\item Find the intervals of values of a for which the line $y+x=0$ bisects two chords drawn from a point  $\sbrak{\frac{1+\sqrt{2}a}{2},\frac{1-\sqrt{2}a}{2}}$ to the circle $2x^2+2y^2-(1+\sqrt{2}a)x+(1-\sqrt{2}a)y=0$. (1996)
\item A circle passes through three points A, B and C with the line segment AC as its diameter. A line passing through A intersects the chord BC at a point D inside the circle. If angles DAB and CAB are $\alpha$ and $\beta$ respectively and the distance betwcen the point A and the mid-point of the line segment DC is d, prove that the area of the circle is $\frac{\pi d^2 \cos^2 \alpha}{\cos^2\alpha+\cos^2\beta+ 2\cos\alpha\cos\beta\cos(\beta-\alpha)}$ (1996)
\item Let C be any circle with centre $(0,\sqrt{2})$. Prove that at the most two rational points can be there on C. (A rational point is a point both of whose coordinates are rational numbers) (1997)
\item $C_1$ and $C_2$ are two concentric circles, the radius of $C_2$  being twice of that $C_1$. From a point P on $C_2$, tangents PA and PB are drawn to $C_1$. Prove that the centroid of the triangle PAB lies on $C_1$.(1998)
\item Let $T_1$,$T_2$ be two tangents drawn from (-2,0) on to the circle $C:x^2+y^2=1$. Determine the circle touching C and having $T_1,T_2$ as their pair of tangents. Further, find the equations of all possible common tangents to these circles, when taken two at a  time (1999)
\item Let $2x^2+y^2-3xy=0$ be the equation of a pair of tangents drawn from the origin O to a circle of radius 3 with centre in the first quadrant. If A is one of the points of contact, find the length of OA.(2001)
\item Let $C_1$ and $C_2$  be two circles with $C_2$ , lying inside $C_1$  A circle  Clying inside $C_1$, touches $C_1$ internally and $C_2$ externally. Identify the locus of the centre of C . (2001)
\item For the circle $x^2+y^2=r^2$, find the value of r for which the area enclosed by the tangents drawn from the point P (6, 8) to the circle and the chord of contact is maximum.(2003)
\item Find the equation of circle touching the line $2x+3y+1=0$ at (1,- 1) and cutting orthogonally the circle having line segment joining (0,3) and (-2,-1) as diameter.(2004)
\item Circles with radii 3,4 and 5 touch each other externally. If P is the point of intersection of tangents to these circles at their points of contact, find the distance of P from the points of contact (2005)

\end{enumerate}


\section*{F  :  Match The Following}
 
 DIRECTIONS (Q.1): Each question contains statements given in two columns, which have to be matched. The statements in Column-I are labelled 1, 2, 3 and 4. while the statements in Columa-II are labelled as a,b,c, d and e. Any given statement in Column-I can have correct matching with ONE OR MORE statement(s) in Column-II. The appropriate bubbles corresponding to the answer to these questions have to be darkened as illustrated in the following example:
If the correct matches are 1-a. s and e: 2-b and c: 3-1 and 2: and 4-d 

\begin{enumerate}
\item Let the circles $C_1:x^2+y^2=9$ and $C_2:(x-3)^2+(y-4)^2=16$, intersect at the points X and Y. Suppose that another circie $C_3:(x-h)^2+(y-ky)^2 =r^2$ satisfies the following conditions:
\begin{enumerate}
\item Centre of $C_3$, is collinear with the centres of $C_1$, and $C_2$
\item $C_1$ and $C_2$  both lie inside $C_3$  and
\item  $C_3$ touches $C_1$ at M and $C_2$ at N
\item Let the line through X and Y intersect $C_3$ at Z and W, and let a common tangent of $C_1$ and $C_2$ be a tangent to the parabola $x^2=8\alpha y$.
There are some expressions given in the List - I whose values are given in List - II below
\\
\end{enumerate}
\begin{multicols}{2}
column-I \\
column-II

\end{multicols}

\begin{matchtabular}
$(2h+k)$ & 6\\
$\frac{Length of ZW}{Length of  XY}$ & $\sqrt{6}$\\
$\frac{Area of triangle MZN}{Area of triangle ZMW}$ & $\frac{5}{4}$\\ & $\frac{21}{5}$\\
& $\sqrt[2]{6}$\\  & $\frac{10}{3}$\\
\end{matchtabular}
Which of the following is the only CORRECT combination?\\
\begin{enumerate}
\item (i)(f)
\item (i)(d)
\item (ii)(e)
\item (ii)(b)
\end{enumerate}

\item Let the circles $C_1:x^2+y^2=9$ and $C_2:(x-3)^2+(y-4)^2=16$, intersect at the points X and Y. Suppose that another circie $C_3:(x-h)^2+(y-ky)^2 =r^2$ satisfies the following conditions:
\begin{enumerate}
\item Centre of $C_3$, is collinear with the centres of $C_1$, and $C_2$
\item $C_1$ and $C_2$  both lie inside $C_3$  and
\item  $C_3$ touches $C_1$ at M and $C_2$ at N
\item Let the line through X and Y intersect $C_3$ at Z and W, and let a common tangent of $C_1$ and $C_2$ be a tangent to the parabola $x^2=8\alpha y$.
There are some expressions given in the List - I whose values are given in List - II below
\\
\end{enumerate}
\begin{multicols}{2}
column-I \\
column-II

\end{multicols}

\begin{matchtabular}
$(2h+k)$ & 6\\
$\frac{Length of ZW}{Length of  XY}$ & $\sqrt{6}$\\
$\frac{Area of triangle MZN}{Area of triangle ZMW}$ & $\frac{5}{4}$\\ & $\frac{21}{5}$\\
& $\sqrt[2]{6}$\\  & $\frac{10}{3}$\\

\end{matchtabular}

Which of the following is the only INCORRECT combination?\\
\begin{enumerate}
\item (iv)(e)
\item (i)(a)
\item (iii)(c)
\item (iv)(f)
\end{enumerate}

\end{enumerate}

\section*{G  :  Comprehension Based Questions}

PASSAGE-1
ABCD is a square of side length 2 units. $C_1$ is the circle touching all the sides of the square ABCD and $C_2$, is the circumcircle of square ABCD. L is a fixed line in the same plane and R is a fixed point.
\begin{enumerate}
\item P is any point of $C_1$, and is another point on $C_2$, then $\frac{PA^2+PB^2+PC^2+PD^2}{QA^2+QB^2+QC^2+QD^2}$ is equal to (2006)
\begin{enumerate}
\item 0.75
\item 1.25
\item 1
\item 0.5
\end{enumerate}
\item Ifa circle is such that it touches the line L and the circle $C_1$ externally, such that both the circles are on the same side of the line, then the locus of centre of the circleis (2006)
\begin{enumerate}
\item ellipse
\item parabola
\item hyperbola
\item pair of straight line
\end{enumerate}
\item A line $L'$ through A is drawn parallel to BD. Point S moves such that its distances from the line BD and the vertex A are equal. 1f locus of S cuts $L'$ at $T_2$ and $T_3$ and AC at $T_1$ then area of $\triangle T_1T_2T_3$ is (2006)
\begin{enumerate}
\item $\frac{1}{2}$ sq.units
\item $\frac{2}{3}$ sq.units
\item 1 sq. units
\item 2 sq. units
\end{enumerate}

PASSAGE-2

A circle C of radius 1 is inscribed in an equilateral triangle PQR. The points of contact of C with the sides PQ. QR, RP are D, E, F,respectively. The line PQ is given by the equation $\sqrt{3}x+y-6=0$ and the point D is $\sbrak{\frac{\sqrt[3]{3}}{2},\frac{3}{2}}$. Further, it is given that the origin  and the centre of C are on the same side of the line PQ. 
\item The equation of circle C is (2008) 
\begin{enumerate}
\item $(x-\sqrt[2]{3})^2+(y-1)^2=1$
\item $(x-\sqrt[2]{3})^2+(y+\frac{1}{2})^2=1$
\item $(x-\sqrt{3}^2+(y+1)^2=1$
\item $(x-\sqrt{3}^2+(y-1)^2=1$
\end{enumerate}
\item Points E and F are given by (2008) 
\begin{enumerate}
\item $\sbrak{\frac{\sqrt{3}}{2},\frac{3}{2}},\sbrak{\sqrt{3},0}$
\item $\sbrak{\frac{\sqrt{3}}{2},\frac{1}{2}},\sbrak{\sqrt{3},0}$
\item $\sbrak{\frac{\sqrt{3}}{2},\frac{3}{2}},\sbrak{\frac{\sqrt{3}}{2},\frac{1}{2}}$
\item $\sbrak{\frac{3}{2},\frac{\sqrt{3}}{2}},\sbrak{\frac{\sqrt{3}}{2},\frac{1}{2}}$
\end{enumerate}
\item Equations of the sides QR, RP are (2008) 
\begin{enumerate}
\item $y=\frac{2}{\sqrt{3}}x+1$,$y=\frac{2}{\sqrt{3}}x-1$
\item $y=\frac{1}{\sqrt{3}}x,y=0$
\item $y=\frac{\sqrt{3}}{2}x+1$,$y=\frac{\sqrt{3}}{2}x-1$
\item $y=\sqrt{3}x,y=0$
\end{enumerate}

PASSAGE-3
A tangent PT is drawn to the circle $x^2+y^2=4$ at the point $P(\sqrt{3},1)$. A straight line L, perpendicular to PTis atangent to the circle $(x-3)^2+y^2=1$.
\item A possible equation of L is (2012)
\begin{enumerate}
\item $x-\sqrt{3}y=1$
\item $x+\sqrt{3}y=-1$
\item $x-\sqrt{3}y=1$
\item $x+\sqrt{3}y=5$
\end{enumerate}
\item A common tangent of the two circles is 
\begin{enumerate}
\item x=4
\item y=2
\item $x+\sqrt{3}y=4$
\item $x+\sqrt[2]{2}y=6$
\end{enumerate}

PASSAGE-4
Let S be the circle in the xy-plane defined by the equation $x^2+y^2=4$
\item Let $E_1,E_2$ and $F_1,F_2$, be the chords of S passing through thepoint $(P_1,P_2)$ and parallel to the x-axis and the y-axis respectively. Let $G_1,G_2$, be the chord of S passing through $P_0$ and having slope -1. Let the tangents to S at $E_1$ and $E_2$ meet at $E_3$, the tangents to S at $F_1$ and $F_2$ meet at $F_3$ and the tangents to S at $G_1$ and $G_2$ meet at $G_3$. Then, the points $E_3$, $F_3$, and $G_3$ lie on the curve (2018)
\begin{enumerate}
\item $x+y=4$
\item $(x-4)^2-(y-4)^2=16$
\item $(x-4)(y-4)=4$
\item $xy=4$
\end{enumerate}
\item Let P be a point on the circle S with both coordinates being positive. Let the tangent to S at P intersect the coordinate axes at the points M and N. Then, the mid-point of the line segment MN must lie on the curve (2018)
\begin{enumerate}
\item $(x+y)^2=3xy$
\item $x^\frac{2}{3}+y^\frac{2}{3}=2^\frac{4}{3}$
\item $x^2+y^2=2xy$
\item $x^2+y^2=x^2y^2$
\end{enumerate}

\end{enumerate}

\section*{H   :  Assertion And Reason Type Questions}

\begin{enumerate}
\item Tangents are drawn from the point (17, 7) to the circle $x^2+y^2=169$
STATEMENT-1: Thetangents are mutually perpendicular because
STATEMENT-2: The locus of the points from which mutually perpendicular tangents can be drawn to the given circle is $x^2+y^2=338$ (2007)
\begin{enumerate}
\item Statement-1 is True, statement-2 is True; Statement-2 is a correct explanation for Statement-1.
\item Statement-1 is True, Statement-2 is True;, Statement-2 is NOT a correct explanation for Statement-1.
\item Statement-1 is True, Statement-2 is False.
\item  Statement-1 is False, Statement-2 is True.
\end{enumerate}
\item Consider $L_1:2x+3y+p-3=0$\\
$L_2:2x+3y+p+3=0$\\
where p is a real number, and $C:x^2+y^2+6x-10y+30=0$
STATEMENT-1:If line $L_1$, is a chord of circle C, then line $L_2$ is not always a diameter of circle C and 
STATEMENT-2:If line $L_1$, is a diameter of circle C, then line $L_1$, is not a chord of circle C. (2008)
\begin{enumerate}
\item Statement-1 is True, Statement-2 is True; Statement-2 is acorrect explanation for Statement-1
\item Statement-1 is True, Statement-2 is True; Statement-2 is NOT a correct explanation for Statement-1
\item Statement-1 is True, Statement-2 is False
\item Statement-1 is False, Statement-2 is True
\end{enumerate}
\end{enumerate}

\section*{I    : Integer Value Correct Type }

\begin{enumerate}
\item The centres of two circles $C_1$ and $C_2$ cach of unit radius are at a distance of 6 units from each other. Let P be the mid  point of the line segement joining the centres of $C_1$ and $C_2$ and C be a circle touching circles $C_1$ and $C_2$ externally. If a common tangent to $C_1$, and C passing through P is also a common tangent to $C_2$ and C, then the radius of the circle C (2009)
\item The straight line $2x-3y=1$ divides the circular region $x^2+y^2\leq 6$ into two parts. If $S=\sbrak{\sbrak{2,\frac{3}{4}},\sbrak{\frac{5}{2},\frac{3}{4}},\sbrak{\frac{1}{4},-\frac{1}{4}},\sbrak{\frac{1}{8},\frac{1}{4}}}$ then the number of points (s) in S lying inside the smaller part is (2011)
\item For how many values of p, the circle $x^2+y^2+2x+4y-p=0$ and the coordinate axes have exactly three common points? (2017)
\item Let the point B be the reflection of the point A(2,3) with respect to the line $8x-6y-23=0$. Let $T_A$ and $T_B$ be circles of radii 2 and 1 with centers Aand B respectively. Let T be a common tangent to the circles $T_A$ and $T_B$ such that both the circles are on the same side of T. If C is the point of intersection of T and the line passing throughA and B, then the length of the line segment AC is (2019)
\end{enumerate}

\section*{Section-B    [JEE Mains /AIEEE]}


\begin{enumerate}
\item If the chord $y=mx+1$ of the circle $x^2+y^2=1$ subtends an angle of measure $45^\circ$ at the major segment of the circle then value of m is (2002)
\begin{enumerate}
\item $2\pm \sqrt{2}$
\item $-2\pm \sqrt{2}$
\item $-1\pm \sqrt{2}$
\item none of these
\end{enumerate}
\item The centres of a set of circles, each of radius 3, lie on the circle $x^2+y^2=25$. The locus of any point in the set is (2002)
\begin{enumerate}
\item $4\leq x^2+y^2\leq 64$
\item $ x^2+y^2\leq 25$
\item $x^2+y^2\geq 25$
\item $3\leq x^2+y^2\leq 9$
\end{enumerate}
\item he centre of the circle passing through (0,0) and (1,0) and touching the circle $x^2+y^2=9$ is (2002)
\begin{enumerate}
\item $\sbrak{\frac{1}{2},\frac{1}{2}}$
\item $\sbrak{\frac{1}{2},-\sqrt{2}}$
\item $\sbrak{\frac{3}{2},\frac{1}{2}}$
\item $\sbrak{\frac{1}{2},\frac{3}{2}}$
\end{enumerate}
\item The equation of a circle with origin as a centre and passing through equilater al triangle whose median is of length 3a is (2002)
\begin{enumerate}
\item $x^2+y^2=9a^2$
\item $x^2+y^2=16a^2$
\item $x^2+y^2=4a^2$
\item $x^2+y^2=a^2$
\end{enumerate}
\item If the two circles $(x-1)^2(y-3)^2=r^2$ and $x^2+y^2-8x+2y+8=0$ intersect in two distinct point, then (2003)
\begin{enumerate}
\item $r>2$
\item $2<r<8$
\item $r<2$
\item $r=2$
\end{enumerate}
\item The lines $2x-3y=5$ and $3x-4y=7$ are diameters of a circle having area as 154 sq.units.Then the equation of the circle is (2003)
\begin{enumerate}
\item $x^2+y^2-2x+2y=62$ 
\item $x^2+y^2+2x-2y=62$
\item $x^2+y^2+2x-2y=47$ 
\item $x^2+y^2-2x+2y=47$
\end{enumerate}
\item If a circle passes through the point (a,b) and cuts the circle $x^2+y^2=4$ orthogonally, then the locus of its centre is (2004)
\begin{enumerate}
\item $2ax-2by-(a^2+b^2+4)=0$
\item $2ax+2by-(a^2+b^2+4)=0$
\item $2ax-2by+(a^2+b^2+4)=0$
\item $2ax+2by+(a^2+b^2+4)=0$
\end{enumerate}
\item A variable circle passes through the fixed point A(p,q) and touches x-axis. The locus of the other end of the diameter through A is (2004)
\begin{enumerate}
\item $(y-q)^2=4px$
\item $(x-q)^2=4py$
\item $(y-p)^2=4qx$
\item $(x-p)^2=4qy$
\end{enumerate}
\item If the lines $2x+3y+1=0$ and $3x-y-4=0$ lie along diameter of a circle of circumference $10\pi$, then the equationofthe circle is (2004)
\begin{enumerate}
\item $x^2+y^2+2x-2y-23=0$
\item $x^2+y^2+2x+2y-23=0$
\item $x^2+y^2+-x-2y-23=0$
\item $x^2+y^2-2x+2y-23=0$
\end{enumerate}
\item Intercept on the line y=x by the circle $x^2+y^2-2x=0$ is AB. Equation of the circle on AB as a diameter is (2004)
\begin{enumerate}
\item  $x^2+y^2+x-y=0$
\item  $x^2+y^2+x+y=0$
\item  $x^2+y^2-x+y=0$
\item  $x^2+y^2-x-y=0$
\end{enumerate}
\item If the circles $x^2+y^2+2ax+cy+a=0$ and $x^2+y^2-3ax+dy-1=0$ intersect in two distinct points P and Q then the line 5x +by-a=0 passes through P and Q for (2005)
\begin{enumerate}
\item exactly one value ofa
\item no value of a
\item infinitely many values of a
\item  exactly two values of a
\end{enumerate}
\item A circle touches the x-axis and also touches the circle with centre at (0,3) and radius 2. The locus of the centre of the circle is  (2005)
\begin{enumerate}
\item an ellipse
\item a circle
\item a hyperbola
\item a parabola
\end{enumerate}
\item  If a circle passes through the point (a,b) and cuts the circle $x^2+y^2=p^2$ orthogonally, then the equation of the locus of its centre is (2005)
\begin{enumerate}
\item $x^2+y^2-3ax-4by+(a^2+b^2-p^2)=0$
\item $2ax+2by-(a^2-b^2+p^2)=0$
\item $x^2+y^2-2ax-3by+(a^2-b^2-p^2)=0$
\item $2ax+2by-(a^2-b^2+p^2)=0$
\end{enumerate}
\item If the pair of lines $ax^2+2(a+b)xy+by^2=0$ lie along diameters ofa circle and divide the circle into four sectors such that the area of one of the sectors is thrice the area of another sector then (2005)
\begin{enumerate}
\item $3a^2-10ab+3b^2=0$ 
\item $3a^2+10ab+3b^2=0$ 
\item $3a^2-2ab+3b^2=0$ 
\item $3a^2+2ab+3b^2=0$ 
\end{enumerate}
\item If the lines 3x-4y -7 = 0 and 2x-3y-5 =0 are two diameters of a circle of area $49\pi$ square units, the equation ofthe circle is (2006)
\begin{enumerate}
\item $x^2+y^2+2x-2y-47=0$
\item $x^2+y^2+2x-2y-62=0$
\item $x^2+y^2-2x+2y-47=0$
\item $x^2+y^2-2x+2y-62=0$
\end{enumerate}
\item Let C be the circle with centre (0,0) and radius 3 units. The equation of the locus of the mid points of the chords of the circle C that subtend an angle of $\frac{2\pi}{3}$ at its center is (2006)
\begin{enumerate}
\item $x^2+y^2=\frac{3}{2}$
\item $x^2+y^2=1$
\item $x^2+y^2=\frac{27}{4}$
\item $x^2+y^2=\frac{9}{4}$
\end{enumerate}
\item  Consider a family of circles which are passing through the point(-1,1) and are tangent to x-axis. If (h,k) are the coordinate of the centre of the circles, then the set of values of k is given by the interval (2006)
\begin{enumerate}
\item $-\frac{1}{2}\leq k\leq \frac{1}{2}$
\item $k\leq \frac{1}{2}$
\item $0\leq k\leq \frac{1}{2}$
\item $k\geq \frac{1}{2}$
\end{enumerate}
\item  The point diametrically opposite to the point P(1,0) on the circle $x^2+y^2+2x+4y-3=0$ is (2007)
\begin{enumerate}
\item (3,-4)
\item (-3,4)
\item (-3,-4)
\item (3,4)
\end{enumerate}
\item The differential equation of the family of circles with fixed radius 5 units and centre on the line $y=2$ is (2008)
\begin{enumerate}
\item $(x-2)y'^2 =25-(y-2)^2$
\item $(y-2)y'^2=25-(y-2)^2$
\item $(y-2)^2y'^2=25-(y-2)^2$
\item $(x-2)^2y'^2 =25-(y-2)^2$
\end{enumerate}
\item If P and Q are the points of intersection of the circle $x^2+y^2+33x+7y+2p-5=0$ and $x^2+y^2+2x+2y-p^2=0$ then there is a circle passing through P,Q and (1,1) for: (2009)
\begin{enumerate}
\item all except one value of p
\item all except two values of p
\item exactly one value of p
\item all values of p 
\end{enumerate}
\item The circle $x^2+y^2=4x+8y+5$ intersects the line $3x-4y=m$ at two distinct points if (2010)
\begin{enumerate}
\item $-35<m<15$
\item $15<m<65$
\item $35<m<85$
\item $-85<m<-35$ 
\end{enumerate}
\item The two circles $x^2+y^2=ax$ and $x^2+y^2=c^2(c> 0)$ touch each other if (2011)
\begin{enumerate}
\item $\mid a\mid=c$
\item $a=2c$
\item $\mid a\mid=2c$
\item $2\mid a\mid=c$ 
\end{enumerate}
\item The length of the diameter of the circle which touches the x-axis at the point (1,0) and passes through the point (2,3) is (2012)
\begin{enumerate}
\item $\frac{10}{3}$
\item $\frac{3}{5}$
\item $\frac{6}{5}$
\item $\frac{5}{3}$
\end{enumerate}
\item The circle passing through (1,-2) and touching the axis ofx at (3,0) also passes through the point (2013)
\begin{enumerate}
\item (-5,2)
\item (5,-2)
\item (-5,-2)
\item (5,2)
\end{enumerate}
\item Let C be the circle with centre at (1,1) and radius =1. If T is the circle centred at (0,y), passing through origin and touching the circle C externally, then the radius of T is equal (2014)
\begin{enumerate}
\item $\frac{1}{2}$
\item $\frac{1}{4}$
\item $\frac{\sqrt{3}}{\sqrt{2}}$
\item $\frac{\sqrt{3}}{2}$
\end{enumerate}
\item Locus of the image of the point (2,3) in the line $(2x-3y +4)+k(x-2y+3)=0$, $k\in R$, is a: (2015)
\begin{enumerate}
\item circle of radius $\sqrt{2}$.
\item circle of radius $\sqrt{3}$.
\item straight line parallel tox-axis.
\item straight line parallel to y-axis.
\end{enumerate}
\item The number of common tangents to the circles $x^2+y^2-4x-6x-12=0$ and $x^2+y^2+6x +18y+26=0$, is:  (2015)
\begin{enumerate}
\item 3
\item 4
\item 1
\item 2
\end{enumerate}
\item The centres of those circles which touch the circle, $x^2+y^2-8x-8y-4=0$, externally and also touch the x-axis,lie on: (2016)
\begin{enumerate}
\item a hyerbola
\item a parabola
\item a circle
\item an ellipse, which is not a circle 
\end{enumerate}
\item If one of the diameters of the circle, given by the equation, $x^2+y^2-4x+6y-12=0$, is a chord of a circle S, whose centre is at (-3,2),then the radius of S is:  (2016)
\begin{enumerate}
\item 5
\item 10
\item $\sqrt[5]{2}$
\item $\sqrt[5]{3}$
\end{enumerate}
\item  Ifa tangent to the circle $x^2+y^2=l$ intersects the coordinate axes at distinct points P and Q, then the locus of the mid-point of PQ is: (2019)
\begin{enumerate}
\item $x^2+y^2-4x^2y^2=0$
\item $x^2+y^2-2xy=0$
\item $x^2+y^2-16x^2y^2=0$
\item $x^2+y^2-2x^2y^2=0$
\end{enumerate}
\end{enumerate}
\end{document}
